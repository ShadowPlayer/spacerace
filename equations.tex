\documentclass[draft=false,fontsize=12pt,paper=a4]{scrartcl}

\usepackage{xltxtra}
\defaultfontfeatures{Ligatures=TeX}
\setmainfont[Numbers={OldStyle,Proportional}]{Garamond Premier Pro}
\KOMAoptions{DIV=last}

\title{Spacerace Equations of Motion}
\author{Lachlan McCalman,Brian Thorne, Alistair Reid, Daniel Steinberg}

\begin{document}
\maketitle

\section{First Pass (no collisions)}

The state of a player can be described instantaneously by the tuple
$(x, v, a, \theta, \omega, \alpha)$, where $x$, $v$ and $a$ are the linear
position, velocity and accelerations, and $\theta, \omega, \alpha$ are
rotational position, velocity and accerelation. 

The control inputs for the ships are two \emph{thrusts}, taking the form of
fixed linear and rotational accelerations. $T_l$ is the linear thrust, and
$T_r$ is the rotational thrust. These are assumed to be constant, although
$T_r$ may change sign instantaneously.

Finally, To ensure the craft have a maximum velocity, we assume space is a
viscous medium, in which linear and rotational movement is retarded
proportional to $v^2$ and $\omega^2$ respectively. The factors of
proportionality are denoted $D_l$ and $D_r$.

The resulting equations of motion are:

\begin{align}
  \dot{u}_x &= v_x \\
  \dot{u}_y &= v_y \\
  \dot{v}_x &= T_l\cos\theta - D_lv_x\sqrt{v_x^2 + v_y^2}\\
  \dot{v}_y &= T_l\sin\theta - D_lv_y\sqrt(v_y^2 + v_y^2}\\
  \dot{\theta} &= \omega \\
  \dot{\omega} &= T_r - D_r\omega^2
\end{align}

\end{document}
